\beginsong{Wo's nur Felsen gibt}[
    wuw={nach einem russischen Nomadenlied}, 
    txt={Werner Helwig (Übersetzung)}, 
    txtjahr={1928}, 
    pfii={42}, 
    bo={426}, 
    kssiv={50}, 
    siru={222},
]

\beginverse
\endverse
\includegraphics[draft=false, width=1\textwidth]{Noten/Lied101.pdf}	

\beginverse
\[Dm]Wo im ew'gen Eis stolz der Kasbek \[A]thront, 
hat im stillen Tal einst mein Ahn ge\[Dm]wohnt.
\[Gm]War ein starker \[Dm]Held, \[Gm]kühn und wohlge\[Dm]mut, 
\[A]starb von feindes Hand jäh in seinem \[Dm]Blut.
\endverse

\beginchorus
Wir sind voller Märchen \[A]und Legenden, 
\[Dm]wir haben Schwielen von den \[Gm]Säbeln an den Händen.
Wir schlafen tags und durch\[Dm]reiten die Nächte, 
\[A]stolz auf die Narben vom \[Dm]letzten Gefechte.
\endchorus

\endsong

\beginscripture{}

Dieses Lied ist ein typisches Beispiel für die Vielzahl russophiler Lieder, die in der bündischen Jugend sehr beliebt waren.
Die Tatsache, dass dieses Lied im Lagerliederbuch des KZ Sachsenhausen auftaucht,
belegt zum einen, dass dort Bündische inhaftiert waren, und ist zum anderen Hinweis auf deren kulturelle Aktivität während ihrer Gefangenschaft.

Der Kasbek ist der drittgrößte Berg Georgiens. Er liegt zwischen dem Schwarzen Meer und dem Kaspischen Meer. Wörtlich bedeutet Kasbek 'Eisgipfel'. 
\endscripture
