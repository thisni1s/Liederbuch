\beginsong{Wenn das Brot, dass wir teilen}[
    txt={Claus-Peter März}, 
    mel={Kurt Grahl}, 
    jahr={1981}, 
    kssiv={311},
    oedt={4},
    schm={MR}
]

% \beginverse\memorize
% Wenn das \[C]Brot, dass wir \[Am]teilen, als \[Dm]Rose \[G]blüht
% und das \[C]Wort, dass wir \[Am]sprechen, als \[D]Lied er\[G]klingt.
% \endverse

\beginverse
\endverse
\includegraphics[draft=false, page=1]{Noten/WennDasBrot.pdf}

\beginverse\memorize
Wenn das \[C]Leid jedes \[Am]Armen uns \[Dm]Christus \[G]zeigt
und die \[C]Not, die wir \[Am]lindern, zur \[D]Freude \[G]wird.
\endverse

\beginchorus
Dann hat \[C]Gott unter \[Em]uns schon sein \[F]Haus ge\[C]baut,
dann wohnt \[Am]er schon in \[Dm]unserer \[G]Welt.
Ja dann \[C]schauen wir \[Em]heut' schon sein \[F]Ange\[G]sicht
in der \[Am]Liebe die \[Dm]alles um\[G]fängt, \[C]
in der \[Am]Liebe die \[Dm]alles um\[C]fängt.
\endchorus

\beginverse
Wenn die ^Hand, die wir ^halten, uns ^selber ^hält
und das ^Kleid, das wir ^schenken, auch ^uns ^bedeckt.
\endverse

\printchorus

\beginverse
Wenn der ^Trost, den wir ^geben, uns ^weiter ^trägt
und der ^Schmerz, den wir ^teilen, zur ^Hoffnung ^wird.
\endverse

\printchorus

\beginverse
Wenn das ^Leid, dass wir ^tragen, den ^Weg uns ^weist
und der ^Tod, den wir ^sterben, vom ^Leben ^singt.
\endverse

\printchorus

\endsong
