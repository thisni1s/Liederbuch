\beginsong{Frühling dringt in den Norden}[
    wuw={mayer (Jürgen Sesselmann), Nerother Wandervogel}, 
    jahr={1980, 5. Strophe: 2017}, 
    kssiv={44}, 
    siru={86}, 
    bo={160},
    tonspur={276}, 
    schm={MR}
]

\beginverse
\endverse
\includegraphics[draft=false, page=1]{Noten/FruehlingDringtInDenNorden.pdf}

\beginverse\memorize
\[Em]Sommer \[D]er\[G]füllt \[D]den \[G]Norden, 
Mücken \[D]sind zur \[C]Plage \[D]nun ge\[Em]worden.
In den Höhen \[G]kreist der Greif,
\[C]Lachse ziehn'n zum Laichen \[G]auf \[C]bis ans Ziel und sterben \[G]drauf.
\[C]Lichter \[G]Tag nicht \[Am]enden \[Em]mag im \[C]Sommer \[D]hoch im \[Em]Norden. \[G]\[D]\[Em]
\endverse

\beginverse
^Herbstzeit ^durch^jagt ^den ^Norden, 
erste ^Nächte sind ^frostig ^kalt ge^worden.
Stürme zerr'n an ^gelbem Laub,\newpage
^reife Früchte prahlen ^bunt. ^Bären schwelgen sich dran ^rund,
^gegen ^Süd die ^Graugans ^zieht zur ^Herbstzeit ^hoch im ^Norden. ^^^
\endverse

\beginverse
^Winter ^be^herrscht ^den ^Norden, 
alle ^Wasser sind ^zu Kris^tall ge^worden.
Wölfe heulen ^fern im Tal.
^Lange Zeit Schneekönig ^Mond ^über'm Land alleine ^trohnt
^wie ein ^Spuk der ^Nordlicht ^Flug im ^Winter ^hoch im ^Norden. ^^^
\endverse

\beginverse
^Füllt ^neu ^der ^Lenz den ^Norden,
sind ^die Blüten ^ihm zu^teilge^worden.
Eis treibt schmelzend ^mit dem Strom.
^Abermals die Vögel ^dann ^künden laut den Frühling ^an.
^Jung durch's ^Grün die ^Elche ^zieh'n, im ^nächsten ^Lenz im ^Norden. ^^^
\endverse

\endsong

\beginscripture{}
\begin{tabularx}{.75\textwidth}{@{}X@{}}
Mayer, mit bürgerlichem Namen Jürgen Sesselmann, ist Teil der Nerother Wandervogel, Orden der Bockreiter. Das Lied stammt aus Mayers Nordamerikazyklus und ist im Herbst 1980 am Yukon River entstanden. Die fünfte Strophe schrieb er erst 2017 dazu.
\end{tabularx}
\endscripture
