\beginsong{Bella Ciao}[
    wuw={nach einem italienischen Partisanenlied},
    jahr={1942},
    txt={Übersetzung: Horst Berner}, 
    pfii={101}, 
    pfiii={30}, 
    tonspur={246}, 
    siru={65}, 
    index={Eines Morgens, in aller Frühe},
    schm={UK},
]

\beginverse
Eines \[Em]Morgens, in aller Frühe,
o bella ciao, bella ciao, bella \[H7]ciao, ciao, ciao,
\lrep eines \[Am]Morgens, in aller \[Em]Frühe
trafen \[H7]wir auf unser’n \[Em]Feind. \rrep 
\endverse

\beginverse
Parti\[Em]sanen, kommt, nehmt mich mit euch,
o bella ciao, bella ciao, bella \[H7]ciao, ciao, ciao,
\lrep Parti\[Am]sanen, kommt, nehmt mich \[Em]mit euch,
denn ich \[H7]fühl', der Tod ist ^nah. \rrep
\endverse

\beginverse
Wenn ich ^sterbe, oh ihr Genossen,
o bella ciao, bella ciao, bella ^ciao, ciao, ciao,
\lrep wenn ich ^sterbe, oh ihr Ge^nossen,
Bringt mich ^dann zur letzten ^Ruh'. \rrep
\endverse

\beginverse
In den ^Schatten der kleinen Blume,
o bella ciao, bella ciao, bella ^ciao, ciao, ciao,
\lrep in den ^Schatten der kleinen ^Blume,
in die ^Berge bringt mich ^dann. \rrep
\endverse

\beginverse
Und die ^Leute, die geh'n vorüber,
o bella ciao, bella ciao, bella ^ciao, ciao, ciao,
\lrep und die ^Leute, die geh'n vo^rüber,
seh'n die ^kleine Blume ^steh'n. \rrep
\endverse

\beginverse
Diese ^Blume, so sagen alle,
o bella ciao, bella ciao, bella ^ciao, ciao, ciao,
\lrep ist die ^Blume des Parti^sanen,
der ^für die Freiheit ^starb. \rrep
\endverse

\endsong

\beginscripture{}
Die Melodie wurde bereits Anfang des 20. Jahrhunderts von italienischen Reispflückerinnen in der Nähe der Stadt Bologna gesungen, die die harte Arbeit unter erbarmungslosen Arbeitgebern beklagten. Nach der antifaschistischen Umdichtung wurde das Lied auch außerhalb Italiens bekannt und häufig übersetzt und wird noch heute von Antifaschisten gesungen.
\endscripture
