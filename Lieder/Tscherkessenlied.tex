\beginsong{Tscherkessenlied}[
    wuw={Erich Scholz (olka)},
    jahr={1920?}
]

\beginverse
Die Steppe \[Em]zittert und es \[Hm]klopfen harte \[Em]Hufe:
auf schnellen \[Hm]Pferden naht ein \[Em]Reiterheer.
Es knallen Peitschen und es \[Hm]gellen unsre \[Em]Rufe
vom Kuban  \[Hm]bis zum Schwarzen \[Em]Meer.
Die harte \[D]Faust umspannt die \[G]kurze Lanze,
zum Stoß be\[D]reit, denn zahlreich sind der \[Hm]Feinde Scharen;
\[Em]abends ruft die \[Hm]Trommel uns zum \[Em]Tanze;
die Nacht ist \[Hm]traumlos, kurz und \[Em]schwer.
\endverse

\beginchorus
\[Em]Heia Ohe Ohe, Heia Ohe
\[Hm]wir sind die Tscher\[Em]kessenarmee
\[Em]Heia Ohe Ohe, Heia Ohe
\[Hm]tapfere Tscher\[Em]kessenarmee
\endchorus

\beginverse
Noch gestern ^jagten wir im ^scharfen Nahge^fechte
dem feigen ^Räuber unsre ^Herden ab.
Doch unser Hauptmann fiel, es ^sank die tapfre ^Rechte,
die manchen ^schon geschickt ins ^Grab.
Doch unsre ^Säbel haben gut ge^schnitten,
wie Hunde ^haben wir das Diebes^pack erschlagen;
^dann sind wir die ^Nacht durch toll ge^ritten;
uns führt ein ^toter Hauptmann ^an.
\endverse

\printchorus

\beginverse
Durch unsre ^Dörfer heulen ^laut die Klage^weiber,
die Trommeln ^dröhnen dumpf zum ^Totentanz;
den Fuß des Scheiterhaufens ^bilden tote ^Leiber,
von Feinden, ^die er selbst be^zwang.
Die Flamme ^lohnt – der Haufen bricht zu^sammen,
zum Rache^zug gilt das Kommando: ^Aufgesessen!
^Kameraden, ^rottet euch zu^sammen,
Tscherkessen^horden reiten ^schnell!
\endverse

\printchorus

\endsong

\beginscripture{}
Dieses Lied wurde laut Karl Schrotz von der Navajo-Gruppe am Kölner Georgplatz gesungen. Das Lied wurde auch von der Gruppe um Friedrich Jung, Karl Roggenbach und Fritz Eckert im Kölner Grüngürtel gesungen.
\endscripture
