\beginsong{Probier‘s mal mit Gemütlichkeit}[
    wuw={Terry Gilkyson (für das Dschungelbuch)}, 
    pfiii={202}, 
    kssiv={107}, 
    siru={192},
    oedt={33},
]

\beginchorus
Probier's mal\[D] mit Gemütlichkeit, mit \[G]Ruhe und Gemütlichkeit,
jagst \[D]du den Alltag \[H7]und die Sorgen \[E7]weg! \[A7]
Und wenn du\[D] stets gemütlich bist und \[G]etwas appetitlich ist,
dann \[D]nimm es dir e\[A7]gal von welchem \[D]Fleck!
\endchorus

\beginverse
Was soll ich \[A7]woanders, wo es mir nicht ge\[D]fällt?
Ich gehe nicht \[A7]fort hier, auch nicht für \[D]Geld!
Die Bienen \[G]summen in der \[Gm]Luft, erfüllen \[D]sie mit Honig\[E7]duft,
und \[Hm7]schaust du unter den \[Hm7]Stein, erblickst du \[Em]Ameisen,
die hier \[Em]gut ge\[A]deih'n, pro\[D]bier mal Zwei, \[H7]Drei, Vier.
denn mit Ge\[Em]mütlichkeit kommt \[A7]auch das Glück zu \[D]dir! \[A7]
Es kommt zu \[D]dir!
\endverse


\beginchorus
Probier's mal \[D]mit Gemütlichkeit, mit \[G]Ruhe und Gemütlichkeit,
ver\[D]treibst du deinen \[H7]ganzen Sorgen\[E7]kram. \[A7]
Und wenn du \[D]stets gemütlich bist und \[G]etwas appetitlich ist,
dann \[D]nimm es dir, e\[A7]gal woher es \[D]kam.
\endchorus
 
\beginverse
Na und pflückst du gern ^Beeren und du piekst dich da^bei,
dann lass dich be^lehren: Schmerz geht bald vor^bei!
Du musst be^scheiden, aber nicht ^gierig im Leben ^sein, sonst tust du dir ^weh, 
du ^bist verletzt und ^zeigst nur drauf, ^darum ^pflücke ^gleich mit dem richt'gen ^Dreh! ^
Denn mit Ge^mütlichkeit kommt ^auch das Glück zu ^dir! ^
Es kommt zu ^dir!
\endverse

\printchorus

\beginverse
Was soll ich ^woanders, wo es mir nicht ge^fällt?
Ich gehe nicht ^fort hier, auch nicht für ^Geld!
Die Bienen ^summen in der ^Luft, erfüllen ^sie mit Honig^duft,
und ^schaust du unter den ^Stein, erblickst du ^Ameisen,
die hier ^gut ge^deih'n, pro^bier mal Zwei, ^Drei, Vier.
denn mit Ge^mütlichkeit kommt ^auch das Glück zu ^dir! ^
Es kommt zu ^dir!
\endverse

\printchorus

\endsong
