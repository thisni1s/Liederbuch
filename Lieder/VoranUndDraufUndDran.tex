\beginsong{Voran und drauf und dran}[
    wuw={Werner Helwig}, 
    jahr={1929}, 
]

\beginverse
\[Dm]Voran und drauf und dran, 
wir folgen \[A]alle Mann für \[Dm]Mann. 
\[Gm]Und es ist schwer, 
\[Dm]so freut's uns sehr, 
wir sind das \[A]rasendeste \[Dm]Heer. 
\endverse

\beginverse
^Ahoi, mein lieber Boy, 
ich glaube, ^du musst auch noch ^mit. 
^Drum halte Schritt, 
^im Höllenritt, 
wir haben einen ^festen ^Tritt. 
\endverse


\beginverse
^Jahu, was machst denn du, 
ich glaub', ^du bist nicht unser ^Mann. 
^Ja aber dann, 
^ja aber dann, 
woll'n wir dich lieber ^lassen ^stahn. 
\endverse

\beginverse
^Hau ab, du alter Sack, 
geh' heim und ^lasse uns all^ein.
^Wir brauchen dich 
^ganz sicher nicht, 
auch ohne dich ^geht's fürchter^lich
\endverse


\endsong

\beginscripture{}
Helwig schrieb dieses Lied auf der Burg Waldeck. Mit äußerst begrenzten Mitteln versuchte eine Gruppe von Jugendlichen und Erwachsenen dort eine Jugendburg zu errichten.
Die empfundene Aufbruchsstimmung, der verwegene Plan und die freiheitliche Lebensauffassung spiegeln sich im Liedtext wieder. 
Trotz, Rebellion und die eigene Abgrenzung gegenüber der Spießerwelt werden hier aufs Kühnste gefeiert. 
Die Anfangszeile könnte an das Lied "Wir sind des Geyers schwarzer Haufen" angelehnt sein.
\endscripture
