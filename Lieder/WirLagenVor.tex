\beginsong{Wir lagen vor Madagaskar}[
    wuw={Just Scheu},
    jahr={1934}
]

\beginverse
Wir l\[C]agen vor Madagaskar und \[G]hatten die \[G7]Pest an \[C]Bord,
in den Kesseln, da faulte das Wasser
und t\[G]äglich ging \[G7]einer über B\[C]ord.
\endverse

\beginchorus
Ahoi, Kameraden, a\[G]hoi, a\[C]hoi!
Leb wohl, kleines Madel, leb w\[G]ohl, leb w\[C]ohl!
Ja, w\[C7]enn das S\[F]chifferklavier an B\[C]ord ertönt,
ja, dann sind die Matrosen so s\[G]till, ja, so s\[G7]till,
weil ein j\[C]eder nach seiner Heimat sich sehnt,
die er g\[G]erne einmal w\[G7]iedersehen w\[C]ill.
\endchorus

\beginverse
Wir l\[C]agen schon 14 Tage, kein W\[G]ind durch die S\[G7]egel uns p\[C]fiff.
Der Durst war die größte Plage, da l\[G]iefen wir \[G7]auf ein R\[C]iff.
\endverse

\printchorus

\beginverse
Der l\[C]ange Hein war der erste, er s\[G]off von dem f\[G7]aulen N\[C]ass.
Die Pest gab ihm das Letzte, und w\[G]ir ihm ein S\[G7]eemannsg\[C]rab.
\endverse

\printchorus

\endsong

\beginscripture{}

Es gibt keine Aufzeichnung von „Wir lagen vor Madagaskar“ vor 1933, es steht in keinem der in Frage kommenden Liederbücher. Daher ist zu vermuten, dass das Lied erst zur Zeit des Nationalsozialismus entstand. Es nimmt zwar Bezug auf Ereignisse von 1905, dies jedoch höchstwahrscheinlich ausgelöst durch ein Buch, das 1936 erscheint, diese Ereignisse zum Thema hat und ein Riesenerfolg wird:  „Tsushima“ von Frank Thiess.

Der Roman schildert Ereignisse während des Russisch-Japanischen Krieges (1904/1905), als das “Zweite russische Pazifikgeschwader“ unter Admiral Roschestwenski wegen dringender Reparaturen vor der Nord-West-Küste von Madagaskar ankern musste. Viele russische Soldaten starben in dem tropischen Klima an Typhus, woran  ein Denkmal in Hell Ville auf  Nosy Be (Insel vor Madagaskar) erinnert.

\endscripture
