\beginsong{Rheinische Vandalen}[
    wuw={Werner Helwig}, 
    jahr={1930?}, 
]

\beginverse
Wir sind die \[Dm]rheinischen Van\[A]dalen, 
auf Waldeck raufen, saufen \[Dm]wir. 
Wir haben uns schon viele \[A]Male, 
dort überhoben an dem \[Dm]Bier. 
\endverse

\beginchorus
\[Dm]Hussa, Ma\[Gm]loni, hei mix Kla\[Dm]vier, 
Krambambu\[Gm/E]loni, \[A]Stimmungs-\[Dm]bier. 
\endchorus

\beginverse
Wir bleiben gern, ^wo's guten Wein ^gibt, 
Whisky und Gin, uns wirft nichts ^hin. 
Wir alle lallen freude^trunken 
und fragen nicht mehr nach dem ^Sinn. 
\endverse

\printchorus

\beginverse
In Spanien ^kämpften wir mit Stie^ren, 
dass die Toreros blass vor ^Neid. 
In Indien stahl’n wir den Fak^iren 
den letzten Lendenschurz vom ^Leib 
\endverse

\printchorus

\beginverse
Und kommt mit ^seinem Klappzy^linder 
dereinst der Knochensense^mann, 
wir aber alle feuer^trunken 
begrüßen ihn mit Fluchen ^dann. 
\endverse

\printchorus

\endsong

\beginscripture{}
Werner Helwig schreibt über dieses Lied: "Heimlich trafen sich diese Gruppen auf der Burg Liebenstein, denn auf der Waldeck, der Stammburg der Nerother, war das bereits ein
Unterfangen mit tödlichen Gefahren, das auch bereits seine Opfer gekostet hatte, da sie von Nazispitzeln bewacht wurden. Bei diesen Zusammenkünften entlastete man sich von dem ungeheuren politischen Druck [...]
durch wüstes Singen von Brechtliedern. Durch Persiflagen nazistischer Gepflogenheiten. Durch wilde, in der Mitternachtsstille herausgebrüllte Herausforderungen. Denn in den Kreisen der Kameraden, die nicht
in die Tarnung schlüpfen konnten, folgte eine Verfolgungswelle der anderen. Meist auf Nimmerwiedersehen. Hier wurde nun mit besonderem Genuß gesungen: Wir sind die rheinischen Vandalen ..."
\endscripture








