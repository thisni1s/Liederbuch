\beginsong{An Rhein und Ruhr marschieren wir}[
    wuw={Edelweißpiraten},
    jahr={1943},
]

\beginverse
{\nolyrics Intro:  }
{\nolyrics \lrep \[Em] \[D] \rrep \rep{4}}
\endverse

\newchords{rheinruhrchords}
\beginverse\memorize[rheinruhrchords]
An \[Em]Rhein und \[C]Ruhr marsch\[G]ieren \[D]wir,
für \[Em]unsere F\[C]reiheit kä\[G]mpfen \[D]wir,
den S\[Em]treifendie\[C]nst, schlagt i\[G]hn ent\[D]zwei,
Ede\[Em]lweiß marsc\[C]hiert, Achtung S\[G]traße \[D]frei
\endverse
 
\beginchorus
Mei\[Em]ster g\[C]ib uns d\[G]ie Pap\[D]iere,
Mei\[Em]ster g\[C]ib uns u\[G]nser G\[D]eld,
de\[Em]nn die \[C]Frauen s\[G]ind uns li\[D]eber,
\[Em]als die Sch\[C]ufterei auf d\[G]ieser \[D]Welt.
\[Em]Unser Edelweißp\[C]iratenlager
\[G]liegt in Österreich auf ei\[D]nem Berg
\[Em]sollte es nur ei\[C]ner wagen,
\[G]zu uns zu kommen a\[D]uf den Berg.
\[Em]Wir werden sie \[C]herunterschlagen
\[G]ob Gestapo oder St\[D]reifendienst,
denn un\[Em]sere Edelw\[C]eißpiraten
\[G]kennen keine feige\[D] List.

\endchorus

\beginverse\replay[rheinruhrchords]
^Höre, w^as wir D^ir jetzt s^agen,
^unsere ^Heimat ist n^icht mehr ^frei,
^schwingt die ^Keule wie in ^alten Z^eiten,
^schlagt HJ., ^SA. den ^Schädel en^tzwei
\endverse
 
\printchorus 

\beginverse\replay[rheinruhrchords]
An ^Rhein und ^Ruhr marsch^ieren ^wir,
für ^Räte-Deu^tschland k^ämpfen^ wir,
die R^eakti^on, schlagt s^ie entz^wei,
\lrep Rot^-Front mar^schiert, Achtung ^Straße^ frei \rrep
\endverse
 
\printchorus

\endsong

\beginscripture{}
Parodien, Versionen und Variationen: "Argonnerwald um Mitternacht" ist ein Soldatenlied aus dem Ersten Weltkrieg, das nach älteren Liedern entstand. "In Kiautschau um Mitternacht" wurde auf mit ähnlichem Text und dieser Melodie in der Marine bereits um 1900 gesungen. Durch einen H .v. Gordon wurde dann angeblich 1915 das Argonnerwaldlied als Pionierlied umgetextet und ist in vielen unterschiedlichen Versionen überliefert.

Aus einem Bericht des RSHA über Edelweißpiraten vom 15.3.1943. Das Lied soll von Edelweißpiraten in Gelsenkirchen gesungen worden sein. Das Lied enthält diverse Anspielungen auf Lieder, die vermutlich von der Hitlerjugend und der SA gesungen worden sind – aber auch Handwerkslieder.

Die erste Strophe bezieht sich auf das SA-Lied "Durch deutsches Land marschieren wir", was auf die Melodie vom "Argonnerwald um Mitternacht" gesungen wurde. Die letzten beiden Strophen parodieren "Hohe Tannen weisen die Sterne", was auf die Melodie von "Wahre Freundschaft soll nicht wanken" gesungen wird. "Meister gib uns die Papiere" taucht in mehreren Liedern fahrender Handwerker auf, so im Wolle-Lied.
\endscripture
