\beginsong{Hohe Tannen}[
    txt={Ringpfadfinder},
    mel={schlesisches Volkslied},
    jahr={1921}
]

\beginverse
Hohe \[D]Tannen weisen die \[A7]Sterne, an der \[D]Isar wild s\[A7]chäumender \[D]Flut
liegt das \[G]Lager auch in weiter \[D]ferne doch du Rübezahl \[A7]hütest es \[D]gut
liegt das \[G]Lager auch in weiter \[D]ferne doch du Rübezahl \[A7]hütest es \[D]gut
\endverse

\beginverse
Hat sich u\[D]ns zu eigen g\[A7]egeben, der die S\[D]agen und M\[A7]ärchen dort \[D]spinnt
und im t\[G]iefsten Waldesl\[D]eben, als ein Riese G\[A7]estalt an\[D]nimmt
und im t\[G]iefsten Waldesl\[D]eben, als ein Riese G\[A7]estalt an\[D]nimmt
\endverse

\beginverse
Komm zu \[D]uns an das flackernde \[A7]Feuer in die \[D]Berge bei \[A7]stürmischer \[D]Nacht
schirm die \[G]Zelte die Heimat die t\[D]eure, komm und halte mit \[A7]uns treue \[D]Wacht
schirm die \[G]Zelte die Heimat die t\[D]eure, komm und halte mit \[A7]uns treue \[D]Wacht
\endverse

\beginverse
Höre Rü\[D]bezahl was wir kl\[A7]agen: Volk und H\[D]eimat sind \[A7]nimmermehr f\[D]rei
Schwing die \[G]Keule wie in alten T\[D]agen, schlage Hader und \[A7]Zwietracht entz\[D]wei
Schwing die \[G]Keule wie in alten T\[D]agen, schlage Hader und \[A7]Zwietracht entz\[D]wei
\endverse

\endsong

\beginscripture{}
Dieses Lied entstannt im Rahmen des dritten Aufstandes in Oberschlesien (1921). Die unmittelbare Ursache war die Ablehnung des britisch-italienischen Gebietaufteilungsvorschlags (Percival-de Marinis-Linie)
durch die propolnische Seite, der drei Viertel Oberschlesiens, darunter alle Industriezentren, bei Deutschland belassen wollte. Auf deutscher Seite beteiligten sich auch viele Angehörige der Pfadfinder- und Jugendbewegung an den Kämpfen.
In ihren Reihen ist dieses Lied entstanden. Später, in der Verbotszeit, wurde das Lied verboten und gehörte zum unerwünschten bündischen Liedgut.
Bei den Edelweißpiraten wurde der Text der letzten Strophe zusätzlich abgeändert zu:
'Schwing die Keule wie in alten Zeiten, schlagt HJ, SA den Schädel entzwei' Siehe auch 'An Rhein und Ruhr marschieren wir'
\endscripture
