\beginsong{Unter den Toren}[
    wuw={olka (Erich Scholz), mac (Erik Martin)}, 
    pfii={29}, 
    pfiii={14}, 
    bo={338}, 
    gruen={162}, 
    kssiv={12}, 
    siru={238},
    oedt={51},
    schm={FB (schnell)}
]

\beginverse
\endverse 
\includegraphics[draft=false, width=1\textwidth]{Noten/Lied090.pdf}

\beginverse
\[Em]Silberne Löffel und \[D]Ketten im Sack legst du \[C]besser beim Schlafen dir 
\[H7]unters Genack. \[Em]Zeig' nichts und \[D]sag nichts, die \[G]Messer sind \[D]stumm 
und zu \[H7]kalt ist die Nacht für Gen\[Em]darmen.
\endverse

\beginchorus
\lrep \[G]He\[D]-jo, ein \[G]Feuerlein \[D]brennt, \[Em]kalt ist es \[H7]für Gen\[Em]darmen. \rrep
\endchorus

\beginverse
^Greif' nach der Flasche, doch ^trink' nicht zu viel deine ^Würfel sind gut,
aber ^falsch ist das Spiel. ^Spuck in die ^Asche und ^schau' lieber ^zu denn zu ^kalt ist die Nacht für Gen^darmen. 
\endverse

\beginchorus
\lrep \[G]He\[D]-jo, ein \[G]Feuerlein \[D]brennt, \[Em]kalt ist es \[H7]für Gen\[Em]darmen. \rrep
\endchorus


\beginverse
^Rückt dir die freundliche ^Schwester zu nah, das ist ^gut für die
Wärme mal ^hier und mal da. ^Keiner im ^Dunkeln ver^rät sein
Ge^sicht und zu ^kalt ist die Nacht für Gen^darmen.
\endverse

\beginchorus
\lrep \[G]He\[D]-jo, ein \[G]Feuerlein \[D]brennt, \[Em]kalt ist es \[H7]für Gen\[Em]darmen. \rrep
\endchorus

\beginverse
^Geh' mit der Nacht, eh der ^Frühnebel steigt, nur das ^Feuer bleibt stumm
und das ^Scheit, das verschweigt. ^Lass' nichts zu^rück und ver^giss', was
du ^sahst, denn die ^Sonne bringt bald die Gen^darmen.
\endverse

\beginchorus
\lrep \[G]He\[D]-jo, das \[G]Feuer ist \[D]aus, \[Em]bald kommen \[H7]die Gen\[Em]darmen. \rrep
\endchorus

\endsong

\beginscripture{} 
Gendarm = Wachmann
\endscripture

