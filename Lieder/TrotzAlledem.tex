\beginsong{Trotz alledem}[
    wuw={Hannes Wader}, 
]

\beginverse
Wir \[C]hofften in den \[Dm]Sechzigern
Trotz \[G7]Pop und Spuk und \[Am]alledem
Es \[C]würde nun den \[Dm]Bonner Herrn scharf \[G]eingeheizt, trotz \[F]alle\[C]dem
\endverse
 
\newchords{trotzchorus}
\beginchorus\memorize[trotzchorus]
Doch nun i\[C]st \[F]es \[C]kalt trotz \[G]alledem
Trotz \[C]S-\[F]P-\[C]D und \[G]alledem
Ein \[C]schnöder, scharfer \[F]Winterwind
Durch\[G7]fröstelt uns trotz \[F]alle\[C]dem
\endchorus
 
\beginverse
{\nolyrics Interlude: \[C] \[F] \[G7] \[F] \[C] \[C]}
\endverse

\beginverse
Auch ^Richter und Mag^nifizenz
Samt ^Polizei und ^alledem
Sie ^pfeifen auf die ^Existenz von ^Freiheit, Recht und ^alle^dem
\endverse
 
\beginchorus\replay[trotzchorus]
Trotz ^all^ed^em und ^alledem,
Trotz ^Grund^ge^setz und ^alledem
Drückt ^man uns mit Be^rufsverbot
Die ^Gurgel zu, trotz ^alle^dem
\endchorus
 
\beginverse
{\nolyrics Interlude: \[C] \[F] \[G7] \[F] \[C] \[C]}
\endverse

\beginverse
Doch ^hat der Staat sich ^nur blamiert
Vor ^aller Welt, trotz ^alledem
Und ^wenn die Presse ^Lügen schmiert,das ^Fernsehn schweigt,
trotz ^alle^dem
\endverse

\beginchorus\replay[trotzchorus]
Trotz ^Miß^traun, ^Angst und ^alledem
Es ^kommt ^da^zu, trotz ^alledem
Dass ^sich die Furcht in ^Widerstand
Ver^wandeln wird trotz ^alle^dem!
\endchorus

\endsong

\beginscripture

Dieses Lied ist eine Parodie auf: „Als Noah aus dem Kasten war“, ein studentisches Trinklied von August Kopisch aus dem Jahre 1824. Es wurde im selben Jahr von Karl Gottlieb Reißiger vertont. Es gibt eine ganze Reihe von Nachdichtungen.
Unter anderem zwei Versionen die im Kontext der Revolution von 1848 von Ferdinand Freiligrath geschrieben wurden.
Diese Version, geschrieben von Hannes Wader, bezieht sich vor allem auf die Versionen von Ferdinand Freiligrath.

\endscripture


