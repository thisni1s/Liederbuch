\beginsong{Time to be sleeping}[wuw={?}, capo={7}]

\beginverse
{\nolyrics Intro \[Am]\[G]\[Am]\[G]\[C]\[G]\[Am]\[G]\[Am]}
\endverse

\beginchorus
\[Am]Hush, Hush it's time to be sleeping
\[G]Hush, Hush dreams come a-creeping
\[Am]Dreams \[G]of p\[C]eace and f\[G]reedom
So \[Am]smile in your \[G]sleep bonny \[Am]baby
\endchorus


\beginverse
\[Am]Once our Valleys were ringing
With the \[G]sounds of our children singing
\[Am]Now s\[G]heep b\[C]leat in the e\[G]vening
And the \[Am]sheilings stand \[G]empty and \[Am]broken
\endverse

\printchorus

\beginverse
\[Am]We stood with our heads bowed in prayer
While fac\[G]tors lay our cottages bare
\[Am]Flames \[G]fir\[C]e clear mo\[G]untain air
And\[Am] many were de\[G]ad in the mo\[Am]rning
\endverse

\printchorus

\beginverse

\[Am]Where stands our proud highland mettle?
Our men o\[G]nce so feared in battle
B\[Am]ut now t\[G]hey \[C]stand hudd\[G]led like cattle
Soo\[Am]n to be shi\[G]pped o'er the \[Am]ocean
\endverse

\printchorus

\beginverse
\[Am]No use pleading or praying
Now gone,\[G]gone all hopes of staying
\[Am]So hush\[G] no\[C]w anchors \[G]away
Don\[Am]'t cry in your \[G]sleep bonny \[Am]baby
\endverse

\printchorus

\beginverse
{\nolyrics Outro \[Am]\[G]\[Am]\[G]\[C]\[G]\[Am]\[G]\[Am]\[Am]\[G]\[Am]}
\endverse


\endsong

\beginscripture{}
"Smile In Your Sleep", manchmal auch als "Hush, Hush, Time To Be Sleeping" (schottisch: "Hush, Hush, Time Tae Be Sleepin") bekannt,
ist ein schottisches Volkslied und Klagelied, das von Jim McLean geschrieben und auf die Melodie des gälischen Liedes
"Chi Mi Na Mòrbheanna" (wörtlich "Ich werde die großen Berge sehen" oder "The Mist Covered Mountain") gesetzt wurde.
Es handelt von den Erfahrungen der Crofters während der Highland Clearances mit Vertreibung und Auswanderung.
"Chi Mi Na Mòrbheanna" (im Englischen gemeinhin bekannt als "The Mist Covered Mountains of Home")
ist ein schottisch-gälisches Lied, das 1856 von dem aus Ballachulish stammenden Highlander John Cameron (Iain Camshroin) geschrieben wurde,
der im Gälischen als Iain Rob und Iain Òg Ruaidh bekannt war. 
\endscripture

