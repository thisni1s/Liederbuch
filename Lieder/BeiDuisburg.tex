\beginsong{Bei Duisburg sind viele gefallen}[wuw={unbekannt}]

\beginverse
Bei \[C]Duisburg sind \[C6]viele ge\[G7]fallen
Bei Duisburg gingen viele verl\[C]or´ n
Da \[C7]waren zwei Rotgar\[F]disten
Die ein\[G7]ander die Treue ge\[C]schwor´ n.
\endverse

\beginverse
Die ^schworen eina^nder die ^Treue
Sie hatten einander so ^lieb
Sollte ^einer von beiden ^fallen
Das der ^andere der Mutter dann ^schrieb.
\endverse

\beginverse
Da ^kam eine feind^liche ^Kugel
Durchbohrte dem einen das ^Herz
Für die ^Eltern war es ein ^Kummer
Für die ^Reichswehr war es ein ^Scherz.
\endverse

\beginverse
Als ^nun die Schlacht ^war zu ^Ende
Und sie kehrten zurück ins ^Quartier
Da ^hat sich so vieles ^verändert
Er nahm ^`nen Bleistift und schrieb`s auf ^Papier.
\endverse

\beginverse
Er ^schrieb es mit ^zitternden ^Händen
Er schrieb es mit Tränen im ^Blick
Euer ^Sohn ist vom „Stahlhelm“ ^erschossen
Liegt bei ^Duisburg und kehrt nicht ^zurück.
\endverse

\beginverse
Als ^es die ^Eltern ^erfahren
Daß ihr Sohn sei gefallen im ^Feld
Da bewei^nten sie sein junges ^Leben
Das er ver^lassen so früh schon die ^Welt.
\endverse

\beginverse
Stahlhelm, ^wir ^schwören dir ^Rache
Für vergossenes Arbeiter^blut
Es ko^mmen die Zeiten der ^Rache
Dann be^zahlt ihr`s mit eigenem ^Blut.
\endverse

\endsong
