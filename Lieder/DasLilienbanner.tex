\beginsong{Das Lilienbanner Wehet}[
    txt={Winfried Nuter (winü)},
    mel={Willie Jahn},
    jahr={1924}
]

\beginverse
Das \[E]Lilien\[H7]banner \[E]wehe komm \[A]Bruder, \[H7]reich´ die \[E]Hand
Und wenn der \[H7]Sturm auch\[E] wehet wir \[A]fahren \[H7]durch das \[E]Land
Wir \[A]fahren auf und \[E]nie\[H7]\[E]der zu guter \[H7]Tat be\[E]reit
hell erklingen unsere Lieder: Gut P\[H7]fad, Allzeit Be\[E]reit
\endverse

\beginverse
Und ^unsere ^weiße ^Lilie erm^ahnt ^uns zur ^Pflicht
daß keiner ^je die Tr^eue zu die^sem ^Zeichen ^bricht
Und ^wie des Lö^wen ^Stärke und wie des ^Adlers ^Flug
so sein auch unsere Werke uns sel^ber nie ge^nug
\endverse

\beginverse
Kommt ^Brüder, ^reicht die H^ände kommt ^Brüder ^haltet ^Schritt
Laßt lohn des ^Herzens ^Brände und ^singet fr^eudig ^mit
Seht ^wie die ^Banner ^wehen im ^jungen Morgen^rot
Wir wollen kämpfend siegen wohl ^über unsere ^Not
\endverse

\endsong

\beginscripture{}
Erstmals 1931 gedruckt, Bundeslied des Deutschen Pfadfinderbundes Mosaik (DPBM)
\endscripture
