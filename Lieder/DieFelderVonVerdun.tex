\beginsong{Die Felder von Verdun}[
    wuw={City Preachers}, 
    jahr={1996}, 
    alb={Lieder gegen den Krieg},
]

\beginchorus
Die \[Em]Felder von Ver\[D]dun, die tragen keine \[Em]Ähren \[D] \[Em]
Dort \[D]blüht nur \[A]roter \[Em]Mohn. \[D] \[Em]
Die Gräber von Ver\[D]dun, wem immer sie geh\[Em]ören \[D] \[Em]
Sind \[D]längst \[A]vergessen \[Em]schon.
\endchorus

\beginverse 
Sie \[E]dachten, sie kämen im Herbst schon zurück
Und \[A]zogen mit Fahnen hin\[E]aus.
Sie dachten, es gäbe für sie einen Sieg
Den \[A]brächten sie bald schon nach \[E]Haus!
Doch auf den \[D]Feldern von Ver\[E]dun
War \[A]alle Hoffnung \[E]hin
Und \[D]Krieg und Sieg und \[E]Not und Tod
ver\[A]loren ihren \[H7]Sinn.
\endverse
 
\printchorus

\beginverse
Sie ^wollten den Krieg noch führen wie einst
Was ^kam, das ahnten sie ^nie.
Doch hatten sie kaum die Marne erreicht
Da ^führte der Krieg schon ^sie.
Und auf den ^Feldern von Ver^dun
War ^alle Hoffnung ^hin
Und ^Krieg und Sieg und ^Not und Tod
Ver^loren ihren ^Sinn.
\endverse

\beginverse 
\textnote{Sprechgesang} Das \[Em]Blut der Sold\[D]aten war rot wie der \[Em]Mohn. \[D] \[Em]
Im \[D]Feuer \[A]verbrannte das \[Em]Gras. \[D] \[Em]
Nur wenige \[D]kamen damals \[Em]davon. \[D] \[Em]
Von denen \[D]keiner \[A]jemals \[Em]vergaß.
\endverse

\beginverse 
Wer ^sagt mir, warum sie gestorben sind.
Warum ^dieses Morden gesch^ah?
Denn wenn man nicht endlich zu fragen beginnt.
Dann ^droht uns erneut die ^Gefahr.
Und wie die ^Felder von Ver^dun,
ist ^dann die ganze ^Welt.
Wenn ^du und ich und ^Jedermann
die ^Frage jetzt nicht ^stellt
\endverse
 
\printchorus

\endsong